%%%%%%%%%%%%%%%%%%%%%%%%%%%%%%%%%%%%%%%%%
% Journal Article
% LaTeX Template
% Version 1.4 (15/5/16)
%
% This template has been downloaded from:
% http://www.LaTeXTemplates.com
%
% Original author:
% Frits Wenneker (http://www.howtotex.com) with extensive modifications by
% Vel (vel@LaTeXTemplates.com)
%
% License:
% CC BY-NC-SA 3.0 (http://creativecommons.org/licenses/by-nc-sa/3.0/)
%
%%%%%%%%%%%%%%%%%%%%%%%%%%%%%%%%%%%%%%%%%

%----------------------------------------------------------------------------------------
%	PACKAGES AND OTHER DOCUMENT CONFIGURATIONS
%----------------------------------------------------------------------------------------

\documentclass[twoside,twocolumn]{article}

\usepackage{blindtext} % Package to generate dummy text throughout this template 
%\usepackage[utf8]{inputenc} % Package for unicode characters
\usepackage[utf8]{inputenc}
\usepackage{amssymb}
\usepackage{newunicodechar}
\newunicodechar{Ɖ}{\DH}

\usepackage[sc]{mathpazo} % Use the Palatino font
\usepackage[T1]{fontenc} % Use 8-bit encoding that has 256 glyphs
\linespread{1.05} % Line spacing - Palatino needs more space between lines
\usepackage{microtype} % Slightly tweak font spacing for aesthetics

\usepackage[english]{babel} % Language hyphenation and typographical rules

\usepackage[hmarginratio=1:1,top=32mm,columnsep=20pt]{geometry} % Document margins
\usepackage[hang, small,labelfont=bf,up,textfont=it,up]{caption} % Custom captions under/above floats in tables or figures
\usepackage{booktabs} % Horizontal rules in tables

\usepackage{lettrine} % The lettrine is the first enlarged letter at the beginning of the text

\usepackage{enumitem} % Customized lists
\setlist[itemize]{noitemsep} % Make itemize lists more compact

\usepackage{abstract} % Allows abstract customization
\renewcommand{\abstractnamefont}{\normalfont\bfseries} % Set the "Abstract" text to bold
\renewcommand{\abstracttextfont}{\normalfont\small\itshape} % Set the abstract itself to small italic text

\usepackage{titlesec} % Allows customization of titles
\renewcommand\thesection{\Roman{section}} % Roman numerals for the sections
\renewcommand\thesubsection{\roman{subsection}} % roman numerals for subsections
\titleformat{\section}[block]{\large\scshape\centering}{\thesection.}{1em}{} % Change the look of the section titles
\titleformat{\subsection}[block]{\large}{\thesubsection.}{1em}{} % Change the look of the section titles

\usepackage{fancyhdr} % Headers and footers
\pagestyle{fancy} % All pages have headers and footers
\fancyhead{} % Blank out the default header
\fancyfoot{} % Blank out the default footer
\fancyhead[C]{Ethereum Classic Library $\bullet$ November 2016 $\bullet$ Vol. I,
No.
1} % Custom header text
\fancyfoot[RO,LE]{\thepage} % Custom footer text

\usepackage{titling} % Customizing the title section

\usepackage[pagebackref]{hyperref} % For hyperlinks in the PDF

%----------------------------------------------------------------------------------------
%	TITLE SECTION
%----------------------------------------------------------------------------------------
\setlength{\droptitle}{-4\baselineskip} % Move the title up

\pretitle{\begin{center}\Huge\bfseries} % Article title formatting
\posttitle{\end{center}} % Article title closing formatting
\title{Collective Decentralized Judgements} % Article title
\author{%
\textsc{Prophet Daniel}\thanks{The author would like to thank the Ethereum Classic community.} \\[1ex] % Your name
\normalsize University of Nicosia \\ % Your institution
\normalsize \href{mailto:prophetdaniel@ethereumclassic.org}{prophetdaniel@ethereumclassic.org} % Your email address
\and % Uncomment if 2 authors are required, duplicate these 4 lines if more
\textsc{Olinga Taeed}%\thanks{Corresponding author} \\[1ex] % Second author's
% name
\normalsize University of Northampton \\ % Second author's institution
\normalsize \href{mailto:olinga.taeed@cceg.org.uk}{olinga.taeed@cceg.org.uk} %
% Second author's email address
}
\date{\today} % Leave empty to omit a date
\renewcommand{\maketitlehookd}{%
\begin{abstract}
\noindent \blindtext
\end{abstract}
}

%----------------------------------------------------------------------------------------

\begin{document}

% Print the title
\maketitle

%----------------------------------------------------------------------------------------
%	ARTICLE CONTENTS
%----------------------------------------------------------------------------------------

\section{Introduction}

\lettrine[nindent=0em,lines=3]{T}he majority of judgements are currently
centralized on the figure of a judge with a supposedly high enough general
consciousness about the society and its dynamics to perform fair resolutions.
From a collective perspective, the society is actually relying on the general
consciousness level of a set of judges in the hope of building and mantaining a
just republic.

A general consciousness level is not the most appropriate knowledge to judge
a case. People are very conscient in one subject and inconscient in another.
That means the consciouness level has to be about the subject being judged
\cite{DanielDAGS2016}.

The power is centralized in not only the figure of the judge but also in the
figure of the defendant. Unfortunately the centralized power in these two
figures represents points of failure, where bribery could be incentivized
to happen.

Recent decentralization trend is a disruptive process triggered by the
decentralized access to information that the internet was able to provide.
Besides the internet, there was the advent of the blockchain which is considered
by some even more disrupting than the internet itself. Both decentralization
and the blockchain trends are powered by the scale economy.

The purpose of this paper is to evaluate how these novel concepts can be applied
to the judgement process\cite{AntonopoulosDAMN2016}.
%------------------------------------------------

\section{Conclusion}

\Blindtext

%Maecenas sed ultricies felis. Sed imperdiet dictum arcu a egestas. 
%\begin{itemize}
%\item Donec dolor arcu, rutrum id molestie in, viverra sed diam
%\item Curabitur feugiat
%\item turpis sed auctor facilisis
%\item arcu eros accumsan lorem, at posuere mi diam sit amet tortor
%\item Fusce fermentum, mi sit amet euismod rutrum
%\item sem lorem molestie diam, iaculis aliquet sapien tortor non nisi
%\item Pellentesque bibendum pretium aliquet
%\end{itemize}
%\blindtext % Dummy text

%Text requiring further explanation\footnote{Example footnote}.

%------------------------------------------------

%\section{Results}

%\begin{table}
%\caption{Example table}
%\centering
%\begin{tabular}{llr}
%\toprule
%\multicolumn{2}{c}{Name} \\
%\cmidrule(r){1-2}
%First name & Last Name & Grade \\
%\midrule
%John & Doe & $7.5$ \\
%Richard & Miles & $2$ \\
%\bottomrule
%\end{tabular}
%\end{table}

%\blindtext % Dummy text

%\begin{equation}
%\label{eq:emc}
%e = mc^2
%\end{equation}

%\blindtext % Dummy text

%------------------------------------------------

%\section{Discussion}

%\subsection{Subsection One}

%A statement requiring citation \cite{Figueredo2009}.
%\blindtext % Dummy text

%\subsection{Subsection Two}

%\blindtext % Dummy text

%----------------------------------------------------------------------------------------
%	REFERENCE LIST
%----------------------------------------------------------------------------------------

\bibliographystyle{abbrv}  
\bibliography{tex/bibliography}

%----------------------------------------------------------------------------------------

\end{document}
