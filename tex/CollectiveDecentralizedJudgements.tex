%%%%%%%%%%%%%%%%%%%%%%%%%%%%%%%%%%%%%%%%%
% Journal Article
% LaTeX Template
% Version 1.4 (15/5/16)
%
% This template has been downloaded from:
% http://www.LaTeXTemplates.com
%
% Original author:
% Frits Wenneker (http://www.howtotex.com) with extensive modifications by
% Vel (vel@LaTeXTemplates.com)
%
% License:
% CC BY-NC-SA 3.0 (http://creativecommons.org/licenses/by-nc-sa/3.0/)
%
%%%%%%%%%%%%%%%%%%%%%%%%%%%%%%%%%%%%%%%%%

%----------------------------------------------------------------------------------------
%	PACKAGES AND OTHER DOCUMENT CONFIGURATIONS
%----------------------------------------------------------------------------------------

\documentclass[twoside,twocolumn]{article}

\usepackage{blindtext} % Package to generate dummy text throughout this template 
%\usepackage[utf8]{inputenc} % Package for unicode characters
\usepackage[utf8]{inputenc}
\usepackage{amssymb}
\usepackage{newunicodechar}
\newunicodechar{Ɖ}{\DH}

\usepackage[sc]{mathpazo} % Use the Palatino font
\usepackage[T1]{fontenc} % Use 8-bit encoding that has 256 glyphs
\linespread{1.05} % Line spacing - Palatino needs more space between lines
\usepackage{microtype} % Slightly tweak font spacing for aesthetics

\usepackage[english]{babel} % Language hyphenation and typographical rules

\usepackage[hmarginratio=1:1,top=32mm,columnsep=20pt]{geometry} % Document margins
\usepackage[hang, small,labelfont=bf,up,textfont=it,up]{caption} % Custom captions under/above floats in tables or figures
\usepackage{booktabs} % Horizontal rules in tables

\usepackage{lettrine} % The lettrine is the first enlarged letter at the beginning of the text

\usepackage{enumitem} % Customized lists
\setlist[itemize]{noitemsep} % Make itemize lists more compact

\usepackage{abstract} % Allows abstract customization
\renewcommand{\abstractnamefont}{\normalfont\bfseries} % Set the "Abstract" text to bold
\renewcommand{\abstracttextfont}{\normalfont\small\itshape} % Set the abstract itself to small italic text

\usepackage{titlesec} % Allows customization of titles
\renewcommand\thesection{\Roman{section}} % Roman numerals for the sections
\renewcommand\thesubsection{\roman{subsection}} % roman numerals for subsections
\titleformat{\section}[block]{\large\scshape\centering}{\thesection.}{1em}{} % Change the look of the section titles
\titleformat{\subsection}[block]{\large}{\thesubsection.}{1em}{} % Change the look of the section titles

\usepackage{fancyhdr} % Headers and footers
\pagestyle{fancy} % All pages have headers and footers
\fancyhead{} % Blank out the default header
\fancyfoot{} % Blank out the default footer
\fancyhead[C]{Ethereum Classic Library $\bullet$ November 2016 $\bullet$ Vol. I,
No.
1} % Custom header text
\fancyfoot[RO,LE]{\thepage} % Custom footer text

\usepackage{titling} % Customizing the title section

\usepackage[pagebackref]{hyperref} % For hyperlinks in the PDF

%----------------------------------------------------------------------------------------
%	TITLE SECTION
%----------------------------------------------------------------------------------------
\setlength{\droptitle}{-4\baselineskip} % Move the title up

\pretitle{\begin{center}\Huge\bfseries} % Article title formatting
\posttitle{\end{center}} % Article title closing formatting
\title{Collective Decentralized Judgements} % Article title
\author{%
\textsc{Prophet Daniel}\thanks{The author would like to thank the Ethereum Classic community.} \\[1ex] % Your name
\normalsize University of Nicosia \\ % Your institution
\normalsize \href{mailto:prophetdaniel@ethereumclassic.org}{prophetdaniel@ethereumclassic.org} % Your email address
\and % Uncomment if 2 authors are required, duplicate these 4 lines if more
%\textsc{Olinga Taeed}\thanks{Corresponding author} \\[1ex] % Second author's
\textsc{Olinga Taeed}\thanks{Director of the \href{www.cceg.org.uk}{Centre for Citizenship, Enterprise
and Governance}. CCEG is a spin out Think Tank from the University of
Northampton.} \\[1ex] % Second author's name
\normalsize University of Northampton \\ % Second author's institution
\normalsize \href{mailto:olinga.taeed@cceg.org.uk}{olinga.taeed@cceg.org.uk} %
% Second author's email address
}
\date{\today} % Leave empty to omit a date
\renewcommand{\maketitlehookd}{%
\begin{abstract}
\noindent \blindtext
\end{abstract}
}

%----------------------------------------------------------------------------------------

\begin{document}

% Print the title
\maketitle

%----------------------------------------------------------------------------------------
%	ARTICLE CONTENTS
%----------------------------------------------------------------------------------------

\section{Introduction}

\lettrine[nindent=0em,lines=3]{I}n general, whenever someone needs to access the
current justice system, there is an overwhelming list of requirements to comply
including payment of fees, proofs collection, audience participation,
negotiation with attorneys among others. Each requirement represent a barrier in
the process that statistically reduces the percentage of addressed community
legal needs. Some of them aim to increase the quality of the judgement process,
others are meant to pay for the related costs.


The majority of judgements are currently centralized and ultimately reliant on
the figure of a judge or bench representing higher authority. The system is
reliant on a system that believes this collective possess sufficient general
consciousness about their society and its dynamics to perform fair resolutions.
From a collective perspective, society is actually relying on the general
consciousness level of a set of judges in the hope of building and maintaining a
just republic. A general consciousness level is not the most appropriate
knowledge to judge a case. People may be conscious  in one subject but not in
another. Consciousness levels are linked to the subject being judged\cite{DanielDAGS2016}.

Power is centralized in not only the figure of the judge but also in the figure
of the defendant, both of which represent points of failure in terms of
corruption, bribery or at the very least undue influence. Recent
decentralization trend is a disruptive process triggered by the decentralized
access to information enabled by the internet. Besides the internet, the advent
of the blockchain with its acclaimed disruption. Both decentralization and the
blockchain trends are powered by the scale economy. There is increasing interest
to apply blockchain to non-financial processes, including the judgement
process\cite{AntonopoulosDAMN2016}.

This has become less fantasy and more real by the combination of non-financial
metrics surrounding sentiment and blockchain\cite{Taghiyeva2016} which
allows for very large applications in communities with their own judiciary. In
particular, for example, the creator of the Muslim Arbitration Court (2007) in
the UK, an alternative dispute resolution system overlayed on UK legal
processes, is also launching an Islamic Cryptocurrency based on Muslim belief
systems in 2017\cite{Taeed2016}. A 2020 target to launch a blockchain based
judiciary may be reasonably achievable. There are, however, some significant
ethical and moral infrastructures that need to be taken into consideration in
design.

We propose the development of a fairer justice system that provides resolutions
for the legal needs of the community regardless of individual's economic or
social means\cite{nla.cat-vn632475}.
%------------------------------------------------

\section{Moral and Ethical Dilemmas}
Paradigm selection for a judicial model inevitably leads to questions around
values based contextualisation of legislative frameworks. The existing paradigm
that education, achievement, practice and experience applies to most cultures
and their reliant frameworks to appoint the judiciary.
Daniel\cite{DanielDAGS2016} argues that to attain a more decentralized and thus
equitable structure individuals are ranked according to a prescribed set of
“fitness” suitability criteria even suggesting “evilness” as a measurable
criteria. One could argue, that just exchanging one set of ranking system based
on historic background, with another more questionable and thus more arguably
arbitrary set, does not safeguard the process. It may not be open to bias, but
it is dependent on a robust metrics system that cannot be gamed.
Antonopoulos\cite{AntonopoulosDAMN2016} seeks to develop a smart contract system
for judiciary that may be implemented digitally. Cross border challenges means
that cultural contexts have to be taken into account which are not so easily
appropriated digitally.

Like all kinds of value based systems, there are tangible and intangible
dimensions. Often religious antecedents have proved crucial to legislative
frameworks but here too the dichotomy exists. Progressive Revelation theories of
belief often talk about spiritual laws and cultural laws with the former being
approximately similar across a variety of countries, but certainly not the
latter. Similarly, in-country laws are unambiguous on ‘hard’ values such as
murder, but more speculative on society value based perceptions of acceptable
behaviour. Putting these into a digital contract are reliant on robust objective
digital value metrics that can provide a moral compass against which one can
enshrine limits.

Establishing a digital values based system adds another layer of complexity.
Daniel’s\cite{DanielDAGS2016} suggestion that certain individuals are given more
voting power than others can be considered diagonally opposed to the concept of
decentralisation as part of a democratization process. One can see the
similarity to older established social metric ranking systems which are proxy
based. They give, for example, a financial value to a homeless person in London
that is markedly different to the value of a person in India, or even other
moral rural areas of the UK. The proxy reliant social metrics have been
discredited for this reason alone, as they negate the idea that all men are born
equal and have an equal vote but not value. The Trump and Brexit phenomena are,
indeed, part of decentralization and anti-authority waves which are against any
kind of inequity in terms of influence. But that is not what Daniel proposes.
Each individual in that system would actually have different voting power
depending on the subject.


\section{Conclusion}

\Blindtext

%Maecenas sed ultricies felis. Sed imperdiet dictum arcu a egestas. 
%\begin{itemize}
%\item Donec dolor arcu, rutrum id molestie in, viverra sed diam
%\item Curabitur feugiat
%\item turpis sed auctor facilisis
%\item arcu eros accumsan lorem, at posuere mi diam sit amet tortor
%\item Fusce fermentum, mi sit amet euismod rutrum
%\item sem lorem molestie diam, iaculis aliquet sapien tortor non nisi
%\item Pellentesque bibendum pretium aliquet
%\end{itemize}
%\blindtext % Dummy text

%Text requiring further explanation\footnote{Example footnote}.

%------------------------------------------------

%\section{Results}

%\begin{table}
%\caption{Example table}
%\centering
%\begin{tabular}{llr}
%\toprule
%\multicolumn{2}{c}{Name} \\
%\cmidrule(r){1-2}
%First name & Last Name & Grade \\
%\midrule
%John & Doe & $7.5$ \\
%Richard & Miles & $2$ \\
%\bottomrule
%\end{tabular}
%\end{table}

%\blindtext % Dummy text

%\begin{equation}
%\label{eq:emc}
%e = mc^2
%\end{equation}

%\blindtext % Dummy text

%------------------------------------------------

%\section{Discussion}

%\subsection{Subsection One}

%A statement requiring citation \cite{Figueredo2009}.
%\blindtext % Dummy text

%\subsection{Subsection Two}

%\blindtext % Dummy text

%----------------------------------------------------------------------------------------
%	REFERENCE LIST
%----------------------------------------------------------------------------------------

\bibliographystyle{abbrv}  
\bibliography{tex/bibliography}

%----------------------------------------------------------------------------------------

\end{document}
